%Problem Set 1 LaTeX report for TDT4200
\documentclass[fontsize=11pt, paper=a4, titlepage]{article}
\usepackage{float} %For forcing position of figures
\usepackage{listings}	%For code in document
\usepackage[usenames,dvipsnames]{color}		%For the SkyBlue background color for lstlistings
\usepackage{mathtools}
\DeclarePairedDelimiter{\ceil}{\lceil}{\rceil}
\usepackage{amsfonts,amsmath,amssymb,amsthm}	%For \mathbb
% \usepackage{caption}	%Dunno yet
\usepackage{todonotes}	%For \todo
\usepackage{tabularx}	%for tablecontents wrapping inside cell, instead of cell breaking page width.
\usepackage{verbatim}
\usepackage{enumerate}	%For getting different types of lists like a) II) and so forth.
\usepackage[margin=2cm]{geometry}
\usepackage[utf8, utf8x]{inputenc}	%For norwegian letters and UTF8 encoding support
\usepackage{lastpage}	%For the command \pageref{lastpage}
\usepackage{fancyhdr}
\rfoot{\thepage\ / \pageref{LastPage}}

\usepackage{tikz}
\usepackage{colortbl}
\usetikzlibrary{calc}
\newcolumntype{W}{!{\smash{\vrule
\@width 4\arrayrulewidth
\@height\dimexpr\ht\@arstrutbox+2pt\relax
\@depth\dimexpr\dp\@arstrutbox+2pt\relax}}}
\makeatother
\definecolor{gray}{rgb}{.7,.7,.7}

\newcommand*\Laplace{\mathop{}\!\mathbin\bigtriangleup}

%\setlength{\parindent}{10ex}

\lstset{ %
language=C,							% choose the language of the code
basicstyle=\ttfamily,			% the size of the fonts that are used for the code
numbers=left,						% where to put the line-numbers
numberstyle=\footnotesize,			% the size of the fonts that are used for the line-numbers
stepnumber=1,						% the step between two line-numbers. If it is 1 each line will be numbered
numbersep=5pt,						% how far the line-numbers are from the code
backgroundcolor=\color{SkyBlue},	% choose the background color. You must add \usepackage{color}
showspaces=false,					% show spaces adding particular underscores
showstringspaces=false,				% underline spaces within strings
showtabs=false,						% show tabs within strings adding particular underscores
frame=single,						% adds a frame around the code
tabsize=4,							% sets default tabsize to 4 spaces
captionpos=b,						% sets the caption-position to bottom
breaklines=true,					% sets automatic line breaking
breakatwhitespace=false,			% sets if automatic breaks should only happen at whitespace
escapeinside={\%*}{*)}				% if you want to add a comment within your code
}

\lstset{literate=
	{á}{{\'a}}1 {é}{{\'e}}1 {í}{{\'i}}1 {ó}{{\'o}}1 {ú}{{\'u}}1
	{Á}{{\'A}}1 {É}{{\'E}}1 {Í}{{\'I}}1 {Ó}{{\'O}}1 {Ú}{{\'U}}1
	{à}{{\`a}}1 {è}{{\'e}}1 {ì}{{\`i}}1 {ò}{{\`o}}1 {ù}{{\`u}}1
	{À}{{\`A}}1 {È}{{\'E}}1 {Ì}{{\`I}}1 {Ò}{{\`O}}1 {Ù}{{\`U}}1
	{ä}{{\"a}}1 {ë}{{\"e}}1 {ï}{{\"i}}1 {ö}{{\"o}}1 {ü}{{\"u}}1
	{Ä}{{\"A}}1 {Ë}{{\"E}}1 {Ï}{{\"I}}1 {Ö}{{\"O}}1 {Ü}{{\"U}}1
	{â}{{\^a}}1 {ê}{{\^e}}1 {î}{{\^i}}1 {ô}{{\^o}}1 {û}{{\^u}}1
	{Â}{{\^A}}1 {Ê}{{\^E}}1 {Î}{{\^I}}1 {Ô}{{\^O}}1 {Û}{{\^U}}1
	{œ}{{\oe}}1 {Œ}{{\OE}}1 {æ}{{\ae}}1 {Æ}{{\AE}}1 {ß}{{\ss}}1
	{ç}{{\c c}}1 {Ç}{{\c C}}1 {ø}{{\o}}1 {å}{{\r a}}1 {Å}{{\r A}}1
	{€}{{\EUR}}1 {£}{{\pounds}}1
}
 %config.tex file in same directory

\begin{document}

\begin{center}

{\huge Problem Set 1}\\[0.5cm]

\textsc{\LARGE TDT4200 -}\\[0.5cm]
\textsc{\large Parallel Computations}\\[1.0cm]

\begin{table}[h]
    \centering
    \begin{tabular}{c}
        \textsc{Christian Chavez}
    \end{tabular}
\end{table}

\end{center}
\vfill
\large{\today}
\clearpage

\section{Theory}
\subsection{Problem 1, General Theory}

\begin{enumerate}
\renewcommand{\theenumi}{\Alph{enumi}}

    \item As the development of IT-hardware has reached the Power-wall,
Memory-wall, and ILP-wall, we struggle putting more circuits on-chip due to
the power-wall, we struggle utilizing the circuits we already have on-chip
due to the memory-wall, and we struggle parallellizing the instructions that
are run on-chip due to the ILP-wall.

Hence, the way forward has so far been to instead turn the utilization of
multi-core processors, so as to bypass the three abovementioned walls, and
yet be able to run more and more complicated and complex programs within a
smaller time-frame.

    \item Flynn's taxonomy describes the processing units running programs
based on how what instructions they can run on which data. This description
will also reflect the programs run on the processors in question, since a
there's no hindrance in running a single-thread sequential program on a
multi-core processor architecture.

Flynn's taxonomy's four classifications are as follows:

    \begin{itemize}

        \item Single Instruction, Single Data (SISD). That the processor (and
architecture) supports the runnning of (and runs) a program that only gives the
processor one set of instructions to be run on one set of data.

        \item Single Instruction, Multiple Data (SIMD). That the processor (and
architecture) supports the running of (and runs) a program that only gives the
processor one set of instructions, which can be utilized (run) on  different
sets of data.

This is much like how a Graphical Processing Unit (GPU) works today with its
threads running the same operations on each their segments of the picture
sent to the screen.

        \item Multiple Instructions, Single Data (MISD). That the processor (and
architecture) supports the running of (and runs) a program that can give the
processor several different sets of instructions, to be run (utilized) on the
same set of data.

        \item Multiple Instructions, Multiple Data (MIMD). That the processor
(and architecture) supports the running of (and runs) a program that can give
the processor several different sets of instructions, which are to be utilized
(run) on different sets of data.

A good example of this would be a digital sound-mixer. It receives data in the
form of different sound-channels, and can potentially simultaneously manipulate
each sound-channel differently. One sound-channel may be amplified, another have
its pitch changed, and a third go through a fourier-transformation for misc.
sound-effects.

        \item Finally, this PS requires us to specify the following ``class'',
which is not an official part of Flynn's taxonomy, namely;

Single Program, Multiple Data (SPMD). The idea of the SPMD is to have a program
which can be run in parallel on multiple cores simultaneously, which makes it
very much alike the MIMD class.

    \end{itemize}

    \item MPI is normally used for distributed memory systems. This is because
MPI, when run on more than one processor core at the time, spawns the same
program off on all its running cores. The parallel behaviour stems from the fact
that all the different running instances of the MPI program run their own
instance of the exact same programming code. What makes this a desirable trait
in parallel programming, is that the MPI standard guarantees each core to be
able to identify itself uniquely among all the others, hence each core can
complete a different part of the overall task, even though they all share the
same codebase for the program they're running.

One core might for example be computing something on one part of a matrix, and
another core might be computing this same thing on a different part of the same
matrix.

This then leads us to why MPI is usually used for distributed memory systems.
Each spawned instance of the program, should ideally have its own processing
core, and hence may not share memory with the other spawned instances of the MPI
program. Certainly not L1 cache memory. In a HPC cluster, it's also often the
case that discrete computers are networked to make up the cluster, creating a
situation where if all the processors on the cluster are used to run the same
MPI program, instances of the program may only be able to communicate via the
networking. Sharing no memory resources with each other, not even harddisks.

\end{enumerate}
\vfill
\large{\today}
\end{document}
