%Problem Set 1 LaTeX report for TDT4200
\documentclass[fontsize=11pt, paper=a4, titlepage]{article}
\usepackage{float} %For forcing position of figures
\usepackage{listings}	%For code in document
\usepackage[usenames,dvipsnames]{color}		%For the SkyBlue background color for lstlistings
\usepackage{mathtools}
\DeclarePairedDelimiter{\ceil}{\lceil}{\rceil}
\usepackage{amsfonts,amsmath,amssymb,amsthm}	%For \mathbb
% \usepackage{caption}	%Dunno yet
\usepackage{todonotes}	%For \todo
\usepackage{tabularx}	%for tablecontents wrapping inside cell, instead of cell breaking page width.
\usepackage{verbatim}
\usepackage{enumerate}	%For getting different types of lists like a) II) and so forth.
\usepackage[margin=2cm]{geometry}
\usepackage[utf8, utf8x]{inputenc}	%For norwegian letters and UTF8 encoding support
\usepackage{lastpage}	%For the command \pageref{lastpage}
\usepackage{fancyhdr}
\rfoot{\thepage\ / \pageref{LastPage}}

\usepackage{tikz}
\usepackage{colortbl}
\usetikzlibrary{calc}
\newcolumntype{W}{!{\smash{\vrule
\@width 4\arrayrulewidth
\@height\dimexpr\ht\@arstrutbox+2pt\relax
\@depth\dimexpr\dp\@arstrutbox+2pt\relax}}}
\makeatother
\definecolor{gray}{rgb}{.7,.7,.7}

\newcommand*\Laplace{\mathop{}\!\mathbin\bigtriangleup}

%\setlength{\parindent}{10ex}

\lstset{ %
language=C,							% choose the language of the code
basicstyle=\ttfamily,			% the size of the fonts that are used for the code
numbers=left,						% where to put the line-numbers
numberstyle=\footnotesize,			% the size of the fonts that are used for the line-numbers
stepnumber=1,						% the step between two line-numbers. If it is 1 each line will be numbered
numbersep=5pt,						% how far the line-numbers are from the code
backgroundcolor=\color{SkyBlue},	% choose the background color. You must add \usepackage{color}
showspaces=false,					% show spaces adding particular underscores
showstringspaces=false,				% underline spaces within strings
showtabs=false,						% show tabs within strings adding particular underscores
frame=single,						% adds a frame around the code
tabsize=4,							% sets default tabsize to 4 spaces
captionpos=b,						% sets the caption-position to bottom
breaklines=true,					% sets automatic line breaking
breakatwhitespace=false,			% sets if automatic breaks should only happen at whitespace
escapeinside={\%*}{*)}				% if you want to add a comment within your code
}

\lstset{literate=
	{á}{{\'a}}1 {é}{{\'e}}1 {í}{{\'i}}1 {ó}{{\'o}}1 {ú}{{\'u}}1
	{Á}{{\'A}}1 {É}{{\'E}}1 {Í}{{\'I}}1 {Ó}{{\'O}}1 {Ú}{{\'U}}1
	{à}{{\`a}}1 {è}{{\'e}}1 {ì}{{\`i}}1 {ò}{{\`o}}1 {ù}{{\`u}}1
	{À}{{\`A}}1 {È}{{\'E}}1 {Ì}{{\`I}}1 {Ò}{{\`O}}1 {Ù}{{\`U}}1
	{ä}{{\"a}}1 {ë}{{\"e}}1 {ï}{{\"i}}1 {ö}{{\"o}}1 {ü}{{\"u}}1
	{Ä}{{\"A}}1 {Ë}{{\"E}}1 {Ï}{{\"I}}1 {Ö}{{\"O}}1 {Ü}{{\"U}}1
	{â}{{\^a}}1 {ê}{{\^e}}1 {î}{{\^i}}1 {ô}{{\^o}}1 {û}{{\^u}}1
	{Â}{{\^A}}1 {Ê}{{\^E}}1 {Î}{{\^I}}1 {Ô}{{\^O}}1 {Û}{{\^U}}1
	{œ}{{\oe}}1 {Œ}{{\OE}}1 {æ}{{\ae}}1 {Æ}{{\AE}}1 {ß}{{\ss}}1
	{ç}{{\c c}}1 {Ç}{{\c C}}1 {ø}{{\o}}1 {å}{{\r a}}1 {Å}{{\r A}}1
	{€}{{\EUR}}1 {£}{{\pounds}}1
}
 %config.tex file in same directory for all reports

\begin{document}

\begin{center}

{\huge Problem Set 6, Theory}\\[0.5cm]

\textsc{\LARGE TDT4200 -}\\[0.5cm]
\textsc{\large Parallel Computations}\\[1.0cm]

\begin{table}[h]
    \centering
    \begin{tabular}{c}
        \textsc{Christian Chavez}
    \end{tabular}
\end{table}

\end{center}
\vfill
\hfill \large{\today}
\clearpage

\section*{Problem 1, OpenCL}
    \begin{enumerate}[a)]

        \item

        \item
        \textit{thread} $==$ \textit{something} \\
        \textit{thread block} $==$ \textit{something} \\
        \textit{local memory} $==$ \textit{something} \\
        \textit{shared memory} $==$ \textit{something} \\

        \item

    \end{enumerate}

\section*{Problem 2, Heterogeneous computing}
    \begin{enumerate}[a)]

        \item $w$ is the total work done. $r$ can then be the work done on the
GPU, and $c$ the work done on the CPU, giving $c = w - r$, $0 < r < w$.

If, and only if, the only difference between the CPU and GPU is the time it
takes to run a computationally identical workset, then the division of labor
that results in the minimum execution time will be $1-\frac{1}{10+1}$ for the
GPU, and $1-\frac{10}{10+1}$ for the CPU.

        \begin{align*}
            T_{CPU}(w) &= T_{GPU}(w) \\
            10c &= 1r \\
            r &= 10(w-r) \\
            11r &= 10w \\
            r &= \frac{10}{11}w
        \end{align*}

        \item
        \begin{align*}
            T_{CPU}(w) &= T_{GPU}(w) + T_{in}(r) + T_{out}(w) \\
            10c &= 1r + 0.1r + 0.2r \\
            10c &= 1.3r \\
            r &= \frac{10}{11.3}w
        \end{align*}

        \item The GPU variable used in previous calculations is now represented
as $r_1$ and $r_2$, since the time it takes to execute on the CPU should be
equal to the time it takes to execute on each of the GPUs: $T_{CPU}(w) =
T_{GPU1}(w) = T_{GPU2}(w)$.

Since $T_{GPU2}(w)$ is twice as fast as $T_{GPU1}(w)$ $(T_{GPU1}(w) = w,
T_{GPU2}(w) = 2w, \Rightarrow 2w = 2*w)$, we can now express $r_1$, through the
$r_2$, and vica versa.

        \begin{align*}
            T_{CPU}(w) &= T_{GPU1}(w) \\
            10c &= r_1 \\
            r_1 &= 10(w - r_1 - r_2) \\
            r_1 &= 10(w - 3r_1) \\
            31r_1 &= 10w \\
            r_1 &= \frac{10}{31}w
        \end{align*}

        \begin{align*}
            T_{CPU}(w) &= T_{GPU2}(w) \\
            10c &= r_2 \\
            r_2 &= 10(w - 1.5r_2) \\
            16r_2 &= 10w \\
            r_2 &= \frac{10}{16}w.
        \end{align*}

        \begin{align*}
            c &= w - \frac{10}{16}w - \frac{10}{31}w \\
            c &= w - \frac{31\times 10}{31\times 16} - \frac{16\times 10}{16\times 31} \\
            c &= w - \frac{310}{496}w - \frac{160}{496}w \\
            c &= w(1-\frac{470}{496}) \\
            c &= \frac{26}{496}w = \frac{13}{248}w
        \end{align*}

    \end{enumerate}

\vfill
\hfill \large{\today}
\end{document}
