%Problem Set 1 LaTeX report for TDT4200
\documentclass[fontsize=11pt, paper=a4, titlepage]{article}
\usepackage{float} %For forcing position of figures
\usepackage{listings}	%For code in document
\usepackage[usenames,dvipsnames]{color}		%For the SkyBlue background color for lstlistings
\usepackage{mathtools}
\DeclarePairedDelimiter{\ceil}{\lceil}{\rceil}
\usepackage{amsfonts,amsmath,amssymb,amsthm}	%For \mathbb
% \usepackage{caption}	%Dunno yet
\usepackage{todonotes}	%For \todo
\usepackage{tabularx}	%for tablecontents wrapping inside cell, instead of cell breaking page width.
\usepackage{verbatim}
\usepackage{enumerate}	%For getting different types of lists like a) II) and so forth.
\usepackage[margin=2cm]{geometry}
\usepackage[utf8, utf8x]{inputenc}	%For norwegian letters and UTF8 encoding support
\usepackage{lastpage}	%For the command \pageref{lastpage}
\usepackage{fancyhdr}
\rfoot{\thepage\ / \pageref{LastPage}}

\usepackage{tikz}
\usepackage{colortbl}
\usetikzlibrary{calc}
\newcolumntype{W}{!{\smash{\vrule
\@width 4\arrayrulewidth
\@height\dimexpr\ht\@arstrutbox+2pt\relax
\@depth\dimexpr\dp\@arstrutbox+2pt\relax}}}
\makeatother
\definecolor{gray}{rgb}{.7,.7,.7}

\newcommand*\Laplace{\mathop{}\!\mathbin\bigtriangleup}

%\setlength{\parindent}{10ex}

\lstset{ %
language=C,							% choose the language of the code
basicstyle=\ttfamily,			% the size of the fonts that are used for the code
numbers=left,						% where to put the line-numbers
numberstyle=\footnotesize,			% the size of the fonts that are used for the line-numbers
stepnumber=1,						% the step between two line-numbers. If it is 1 each line will be numbered
numbersep=5pt,						% how far the line-numbers are from the code
backgroundcolor=\color{SkyBlue},	% choose the background color. You must add \usepackage{color}
showspaces=false,					% show spaces adding particular underscores
showstringspaces=false,				% underline spaces within strings
showtabs=false,						% show tabs within strings adding particular underscores
frame=single,						% adds a frame around the code
tabsize=4,							% sets default tabsize to 4 spaces
captionpos=b,						% sets the caption-position to bottom
breaklines=true,					% sets automatic line breaking
breakatwhitespace=false,			% sets if automatic breaks should only happen at whitespace
escapeinside={\%*}{*)}				% if you want to add a comment within your code
}

\lstset{literate=
	{á}{{\'a}}1 {é}{{\'e}}1 {í}{{\'i}}1 {ó}{{\'o}}1 {ú}{{\'u}}1
	{Á}{{\'A}}1 {É}{{\'E}}1 {Í}{{\'I}}1 {Ó}{{\'O}}1 {Ú}{{\'U}}1
	{à}{{\`a}}1 {è}{{\'e}}1 {ì}{{\`i}}1 {ò}{{\`o}}1 {ù}{{\`u}}1
	{À}{{\`A}}1 {È}{{\'E}}1 {Ì}{{\`I}}1 {Ò}{{\`O}}1 {Ù}{{\`U}}1
	{ä}{{\"a}}1 {ë}{{\"e}}1 {ï}{{\"i}}1 {ö}{{\"o}}1 {ü}{{\"u}}1
	{Ä}{{\"A}}1 {Ë}{{\"E}}1 {Ï}{{\"I}}1 {Ö}{{\"O}}1 {Ü}{{\"U}}1
	{â}{{\^a}}1 {ê}{{\^e}}1 {î}{{\^i}}1 {ô}{{\^o}}1 {û}{{\^u}}1
	{Â}{{\^A}}1 {Ê}{{\^E}}1 {Î}{{\^I}}1 {Ô}{{\^O}}1 {Û}{{\^U}}1
	{œ}{{\oe}}1 {Œ}{{\OE}}1 {æ}{{\ae}}1 {Æ}{{\AE}}1 {ß}{{\ss}}1
	{ç}{{\c c}}1 {Ç}{{\c C}}1 {ø}{{\o}}1 {å}{{\r a}}1 {Å}{{\r A}}1
	{€}{{\EUR}}1 {£}{{\pounds}}1
}
 %config.tex file in same directory for all reports

\begin{document}

\begin{center}

{\huge Problem Set 5, Theory}\\[0.5cm]

\textsc{\LARGE TDT4200 -}\\[0.5cm]
\textsc{\large Parallel Computations}\\[1.0cm]

\begin{table}[h]
    \centering
    \begin{tabular}{c}
        \textsc{Christian Chavez}
    \end{tabular}
\end{table}

\end{center}
\vfill
\hfill \large{\today}
\clearpage

\section*{Problem 1, CUDA}
\begin{enumerate}[a)]

    \item
    \begin{enumerate}[I)]

        \item \textbf{CPU}: \\
A \textbf{C}entral \textbf{P}rocessing \textbf{U}nit (CPU), is (as the
name suggest) the main heavy-weight/heavy-duty versatile processing unit which
does not shy away from using a lot of power (Wattage) for completing its
calculations. CPUs also feature complicated (and power hungry) architectural
optimizations like pipelining, cache prediction, and out-of-order execution.

        \item \textbf{GPU}: \\
A \textbf{G}raphics \textbf{P}rocessing \textbf{U}nit (GPU), is a processing
unit meant for calculating, drawing, and rendering images at a high rate per
second. Flaws in the image can be more easily forgiven due to the high density
of pixels in a picture, which results in erroneous pixels not being very visible
to the naked human eye. \\

This makes the GPU more single-focused in its use. There are many applications
where the GPU will outperform a CPU, especially with regards to energy
efficiency, but a modern CPU is much more versatile than a GPU, allowing it to
accomplish a much broader range of applications so much faster that in some
cases it will win on energy-efficiency just because it finished the application
so much quicker. \\

Hence, its strength is focused on the throughput of the device, more so than on
the CPU, where the the accuracy of the results counts as well.

\end{enumerate}

    \item Occupancy in CUDA is a metric for how effectively the \textbf{Grids}
(which consist of \textbf{Blocks}, which in turn consist of \textit{Warps}), are
utilized in the CUDA processor: $\text{Occupancy} = \frac{\text{Active
Warps}}{\text{Maximum Active Warps}}$.

    \item A CUDA \textbf{Block} is a 2D subsection of a \textbf{Grid}, which in
turn consists of \textit{Warps}. Each Warp in turn consists of 32 threads, the
smallest SIMT. \\

There's additional differences between \textbf{Blocks} and \textit{Warps}, with
regards to the memory resources they have, and how these are used/controlled.

    \item Local memory is memory local to one specific thread. This memory is
partitioned out of the general device memory, making it slow since it's
accessible by any thread in the entire \textbf{Grid}. \\

Shared memory however is only accessible by all threads in the same
\textbf{Block}. This alleviates some of the memory access control issues,
permitting operations on this memory to be faster than operations done on local
memory. \\

Local memory is only used when
    \begin{itemize}
        \item Datasize is not known at compile time.
        \item Datasize will exceed available register space.
        \item When register spilling occurs.
    \end{itemize}

\end{enumerate}

\section*{Problem 2, Memory transfers}
Assumption \textbf{1}: Transfer of data only necessary to GPU. Hence, total
execution time for CPU is $kn^2 = 10n^2$. \\
Assumption \textbf{2}: Data must be transferred both \textit{to} and
\textit{from} the GPU, resulting in total execution time to become: $2rn + kn^2
= 2rn + n^2 = n(2r+n)$.

\begin{align*}
    10n^2 &> n(2r+n) \\
    10n &> 2r+n \\
    9n &> 2r \\
    n &> \frac{2r}{9}
\end{align*}

For it to be faster to use the GPU, $n$ must be larger than $2r/9$.

\section*{Problem 3, Thread divergence}
\begin{enumerate}[a)]

    \item If the call to
\begin{lstlisting}[numbers=none]
__syncthreads();
\end{lstlisting}
is not reached by all threads in the same \textbf{Block}, the behavior is
undefined, meaning that it is just as likely that it will deadlock, as it might
crash, as it might return successfully, but with erroneous results. \\

The CUDA documentation states that it is possible to put these calls inside
conditionals, as long as it's guaranteed that all threads in a \textbf{Block}
will reach said call.

    \item The code in example II) will have the best performance.

    \begin{enumerate}[I)]

        \item This code example will be slower to run, since the code is
executed by 32 threads simultaneously in a \textit{Warp}. And since a
\textit{Warp} consists of 32 threads, only $\frac{1}{4}$th of the threads will
run at a time. The threads satisfying the first conditional will run first, with
the remaining $\frac{3}{4}$ of the threads will stand by, idling. This will
repeat for each conditional in the first code example.

        \item This code example will be faster to run, since the threads all
have the same value in their \begin{lstlisting}[numbers=none]
blockIdx.x
\end{lstlisting}
attributes. Hence, it will run faster since all threads in a block will start at
the same time, and run concurrently, as opposed to the previous code example.

    \end{enumerate}
\end{enumerate}

\section*{Problem 4, Memory coalescing}

The first code snippet (\textit{I}) will be the fastest code snippet, due to
this code snippet utilizing memory coalescing. \\

This because it makes each thread retrieve one element per row in the $n\times
n$ data array, whilst the other makes each thread retrieve one whole row by
itself, one thread after another. \\

Memory Coalescing being that the threads retrieve data from consecutive
locations in memory. AKA that thread $1$ retrieves the next data object after
thread $0$, and so forth for a subset of the threads.

\vfill
\hfill \large{\today}
\end{document}
